\documentclass[a4paper,11pt]{article}
\usepackage{ulem}
\usepackage{a4wide}
\usepackage[dvipsnames,svgnames]{xcolor}
\usepackage[pdftex]{graphicx}
\title{Introductory Programming in Python: Flow Control: Functions}
\usepackage[utf8]{inputenc}
\usepackage{hyperref}
% commands generated by html2latex


\begin{document}

\section{Introductory Programming in Python: 
\\    Problem Solving Session}    [\href{functions.html}{Prev: Flow Control: Functions}]\nolinebreak[\href{index.html}{Course Outline}]\nolinebreak 
%  [<a href="scope.html">Next: Variable Scope</a>]


So now we have taught you the basics of python, it is time for you to use what  you've learned to solve some problems. The problems you don't finish will become homework that  you should send in along with your test. Do not try and rush with the problems. An accurate solution  shows what you learn, whereas one that is written quickly and doesn't work means nothing

\section{Sum of multiples}

If we list all the natural numbers below 10 that are multiples of 3 or 5, we get 3, 5, 6 and 9.  The sum of these multiples is 23. 
\\
\\  Write a program that inputs a value N, then calculates and prints the sum of all the multiples of 3 or 5 for  the numbers that are less than N.

\section{Accountants Calculator}

Write an accountants calculator. The user may enter a number, an operator (+, -, *, /), a blank line, or the word 'quit'. The first entry must be a number. Every time a number is entered, it is added to the current total (which    starts at 0), unless the previous line was an operator, in which case, instead of adding, use the operator given to combine the number entered and the total to form a new total. Every time a blank line is entered, print a line of dashes followed by    a line containing the current total. If the entry is the word 'quit' the program ends. Here is an example of output for the input; 4, 9, blank line, *, 2, -, 6, /, 10, blank line, quit     
\begin{lstlisting}

4
9

-----
13
*
2
-
6
/
10

-----
2
quit
\end{lstlisting}

\section{Sets}

A set in mathematics is a collection of distinct elements. The union of two sets A and B consists of all unique elements which are either in set A, or set B, or both set A and set B. The intersection of two sets A and B consists of all elements which are in both set A and set B.

Write a program that reads in two sets of integers, A and B, and calculates both the union and intersection of A and B. The new sets must be output in ascending order, i.e. smallest integers first.

Your program should look similar to the following
\begin{lstlisting}

Enter the size of set A: \textbf{3}
Enter element 1 of set A: \textbf{1}
Enter element 2 of set A: \textbf{2}
Enter element 3 of set A: \textbf{4}

Enter the size of set B: \textbf{4}
Enter element 1 of set B: \textbf{2}
Enter element 2 of set B: \textbf{4}
Enter element 3 of set B: \textbf{6}
Enter element 4 of set B: \textbf{8}

A union B: 1 2 4 6 8
A intersect B: 2 4
\end{lstlisting}

\section{Decompression}

To save time in the transmission of data, it is often compressed. One simple way of doing this is to encode lengthy sequences of the same value by using a single instance of that value followed by the number of times it is repeated. For example if you have a sequence 2222227777 it would be represented as 2 6 7 4.

Your task is to write a program that will receive  A and B. The new sets must be output in ascending as input a sequence of numbers representing encoded data and decompress it. That is, print out the original sequence in its long form.

\section{Hidden Strings}

Strings are just a series of characters 'strung' or joined together. Substrings are strings that are, in fact, just a part of a larger string. One might, for various reasons, wish to find if a string is merely a substring of another string, sometimes disregarding such things as case (UPPER and lower) or punctuation.

Write a program that reads in two sets of integers, A and B, and calculates both the union and intersection of A and B. The new sets must be output in ascending order, i.e. smallest integers first.

    [\href{functions.html}{Prev: Flow Control: Functions}]\nolinebreak[\href{index.html}{Course Outline}]\nolinebreak 
%  [<a href="scope.html">Next: Variable Scope</a>]
   Copyright \copyright James Dominy 2007-2008; Released under the \href{http://www.gnu.org/copyleft/fdl.html}{GNU Free Documentation License}
\\\href{intropython.tar.gz}{Download the tarball}

\end{document}

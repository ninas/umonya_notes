\section{The while statement}

The last program we wrote, to count the number of even and negative   numbers entered by the user, was already getting pretty large if one   measures program size in lines of code. And it only handled three   inputs! What if we wanted a hundred inputs.  Cutting and pasting would   work, but making those small changes to each section of code dealing   with a specific number would be tedious at best, and error prone at   worst. Fortunately, computers excel at repetitive tasks.     Enter the \texttt{while} statement:

\begin{quotation}The \texttt{while} statement executes a statement, or block of statements,   repeatedly, as long as a given expression is \texttt{True}. \end{quotation}

So let's rethink our   previous problem. Previously, we would have described the problem as:
\begin{quotation}     The user enters three integers. The program outputs how many of    those integers were even, and how many were both even and negative.    Also, if the number entered is even, a message stating the number    entered is even is printed. Similarly a message is printed if the    number is both even and negative, indicating the number is    negative.    
\end{quotation}

But we are not happy with just three numbers, so let's re-describe   the behaviour we want from our program
\begin{quotation}     The user may \textbf{enter} a number.    \textbf{Until} the user enters a blank line, the program    continues accepting numbers. After a blank line has been entered    the program \textbf{outputs} the number of numbers entered    by the user \textbf{which} were even, as well as the number    of numbers \textbf{which} were \textbf{both} even \textbf{and}    negative.    
\end{quotation}

Much like how in junior school we were taught to look for key words   in word problems to help us formulate the problem mathematically, we   can and should do the same with descriptions of problems and their   translation into program code. Everything in programming comes down to   one of three \textit{structures}\ldots
\begin{itemize}
	\item A sequence of steps
	\item A condition that selects which sequence of steps to    execute
	\item A repetition of a sequence of steps
\end{itemize}

In the problem description above, we can identify some key words   already hint at the structures we should use.
\begin{description}
	\item[enter] Relates to input -- \texttt{input()}
	\item[until] Indicates \textbf{repetition} based on the fulfilment    of a \textbf{condition} -- \texttt{while} 
	\item[output] Indicates \textbf{output} -- \texttt{print()}
	\item[which] Indicates a \textbf{condition without repetition} -- \texttt{if}    
        \item[both \ldots\ and \ldots] Indicates a \textbf{composition of conditions} -- \texttt{and}    
\end{description}

We'll now step through the problem description, sentence by sentence   and convert it into a Python program, using the keywords as hints, and   the tools we have already come across. Also we'll introduce the syntax   of the while statement.
\begin{quotation}    The user may enter a number.   
\end{quotation}

Looking at the keywords and their hints, we want to use the   \texttt{input} function here. This is true of any input from the keyboard generally.   Thus, our first line of code will be
\begin{lstlisting}
number = input("Enter a number: ")
\end{lstlisting}
\begin{quotation}    Until the user enters a blank line, the program continues accepting numbers.   
\end{quotation}

Okay, from the hints we see we should be using a \texttt{while} statement,   but \textbf{beware}! We don't have an `until' statement in   Python, and in English until and while are opposites, so we just need   to transform the problem slightly so that we now have
\begin{quotation}\textbf{While} the user enters a line \textbf{that is not    blank}, the program continues accepting numbers.    
\end{quotation}

This translates directly into Python code which we append to what   we've already got to produce
\begin{lstlisting}
number = input("Enter a number: ")
while number != '':
    number = input("Enter another number (or nothing to finish): ")
\end{lstlisting}

Note the prompt has changed to tell the user what to do to exit the   program, namely enter a blank line rather than a number. Also note the   format, or syntax, of the \texttt{while} statement.
\begin{lstlisting}
while <expression>:
    statement
    statement
    ...
\end{lstlisting}

See how we \textit{indent} statements we want executed repetitively and   conditionally under the \texttt{while} statement, in the same way as we do with   \texttt{if} statements. The \texttt{while} statement checks the expression, and if it is   \texttt{True}, will execute those indented statements once, after which it will   check the expression again, and execute the statements again, etc \ldots\ If the   expression is ever \texttt{False}, execution of the program continues at the   first un-indented statement after the indented block.

There's a little more to the while loop than pure syntax. Every   loop needs three things; A start point, an end point, and a way to get   from one point to another. More importantly there needs to be a   relationship between these three things, in the form of a variable. The   start point takes the form of assigning a value (the starting value) to   a variable, which we'll call the counter. The while loop's condition   specifies the stop point of the loop, by specifying a condition under   which the loop should terminate \textbf{in terms of the counter   variable}. Finally, the loop needs a way to get from start to   finish in a stepwise manner, i.e. a way to take a single step. This   means the value of the counter variable must change   \textbf{inside} the repeated block of statements, otherwise   value of the loop condition won't change, and the loop will repeat   forever. Commonly, the statement that changes the counter's   value is placed at the \textit{end} of the repeated block of   statements, because this means it is changed immediately before the   counter variable is checked in the loop condition again. This gives us   the pattern:
\begin{lstlisting}
counter = <start_value> #Initialise the counter
while <expression>: #<expression> specifies when the loop will stop in terms of counter
    statement
    statement
    \
    counter = <new value of counter> #changes the value of counter
\end{lstlisting}
\begin{quotation}     After a blank line has been entered the program outputs the total amount    of numbers entered by the user which were even.    
\end{quotation}

Now we haven't kept a record of the numbers entered so how can we   tell how many were even? The solution is to keep a count as even   numbers are entered. How do we store a value?   \textbf{Variables}! We need to know \textbf{how many}   numbers were even, indicating quantity, indicating a number, i.e. an   integer. So let's create a new integer to use whilst counting, and call   it \texttt{even}. But to create a new variable we have to give it a value!   What value can we give even, if we don't know how many even numbers the   user will enter? Well we do know how many even numbers there are before   the user has entered any numbers; there are 0 even numbers. So let's   put an assignment statement to that effect into our program in the   right place (before a number is entered).
\begin{lstlisting}
even = 0
number = input("Enter a number: ")
while number != '':
    number = input("Enter another number (or nothing to finish): ")
\end{lstlisting}

Now at the end of our program we still have 0 even numbers, because   we haven't changed the value of \texttt{even}. We wish to count how many   numbers that are entered are even, which means we need to increment   (increase by one) every time an even number is entered. Firstly, how do   we distinguish even numbers from odd? We use an \texttt{if} statement, because   this is a condition. Secondly, where to we actually place the counting   statements? Here's a suggestion:
\begin{lstlisting}
even = 0
number = input("Enter a number: ")
while number != '':
    if int(number) % 2 == 0: #if number is even
        even = even + 1
    number = input("Enter another number (or nothing to finish): ")
\end{lstlisting}

A few things about our two new lines deserve mention. Firstly, as we   are already familiar with, \texttt{input()} returns a string, so we need to   \textbf{type cast} to an integer before we can test whether it is even. Then   there's the condition itself. The definition of an even number is a   number divisible exactly by 2, i.e. without remainder. The `\texttt{\%}' operator   returns the remainder of a division, and is thus perfectly suited to   the job of testing whether a number is even or odd. If the remainder of   the division is 0, the number is even. Also, we have nested the \texttt{if}   statement within the \texttt{while} statement, so it may be executed multiple   times, once each time a number is entered, to test the number. We have   put it before the input function within the \texttt{while}   statement because we don't want to test an empty string (blank line)   for evenness in the case where the user has not entered a number and   wishes to finish up. This way the \texttt{while} statement's expression tests   whether the input is a blank line, before we convert to an integer and   test for evenness.
\begin{quotation}    as well as the number of numbers which were both even and negative.   
\end{quotation}

The last piece of the problem description says we should also count   how many of the input numbers are negative as well as even. And we need   to put some output at the end once we've counted everything. Well, this   is very similar to the previous segment, so let's recycle the idea and   see what we get\ldots
\begin{lstlisting}
even = 0
negative = 0
number = input("Enter a number: ")
while number != '':
    if int(number) % 2 == 0: #if number is even
        even = even + 1
        if int(number) < 0:
            negative = negative + 1
    number = input("Enter another number (or nothing to finish): ")
print("There were", even, "even numbers, of which", negative, "were also negative")
\end{lstlisting}

So we've included a new variable, \texttt{negative}, which counts the   number of negative entries, but only if those entries are even. Why is   this? We haven't specified (
\texttt{int(number) \% 2 == 0 and int(number)   $<$ 0}). Instead we have nested the test for negativity inside   the test for evenness. This means a number will only be tested, and   thus potentially counted, for negativity if it has already been found   to be even.

\section{The break statement}

Looking at the \texttt{while} statement, it seems that once we're in a block   of statements to be executed repeatedly, known as a   \textbf{loop}, we can't get out of the loop except when the   expression after \texttt{while} (known as the \textbf{loop condition}) is \texttt{False}.   Python provides us with a statement for breaking out of a loop,   conveniently called \texttt{break}. Suppose we   wanted our program for counting even numbers to end not only when a   blank line was entered, but also if the user enters the string \texttt{`quit'}.   We could simply add two lines \ldots
\begin{lstlisting}
even = 0
negative = 0
number = input("Enter a number (or nothing or `quit' to quit): ")
while number != '':
    if number == 'quit':
        break
    if int(number) % 2 == 0: #if number is even
        even = even + 1
        if int(number) < 0:
            negative = negative + 1
    number = input("Enter another number (or nothing to finish): ")
print("There were", even, "even numbers, of which", negative, "were also negative")
\end{lstlisting}

Again we check what the user has entered, and if it is the string   \texttt{`quit'}, we \texttt{break} out of the loop, meaning execution   continues at the \texttt{print()} function.

\section{The continue statement}

If we wanted to get picky and consider 0 to not be even, we would   have to modify our code so that it doesn't count a `0' entry as even or   negative. We can't simply break out of the loop, because the user may   want to enter more numbers after the `0'. We could enclose the entire   test for evenness in an \texttt{if} statement that makes sure the number entered   is not 0, but Python provides us with a more elegant solution, the \texttt{continue} statement.   The \texttt{continue} statement jumps the flow of execution immediately back to   the loop condition, at which point normal loop execution flow   resumes.
\begin{lstlisting}
even = 0
negative = 0
number = input("Enter a number (or nothing or 'quit' to quit): ")
while number != '':
    if number == 'quit':
        break
    if number == '0':
        number = input("Enter another number (or nothing or `quit' to quit): ")
        continue
    if int(number) % 2 == 0: #if number is even
        even = even + 1
        if int(number) < 0:
            negative = negative + 1
    number = input("Enter another number (or nothing or `quit' to quit): ")
print("There were", even, "even numbers, of which", negative, "were also negative")
\end{lstlisting}

\section{else clauses in while loops}

\texttt{while} loop statements may also have an else clause, which is   executed when the loop terminates when the condition becomes \texttt{False}, but   not when the loop is terminated by a \texttt{break} statement.    
\begin{lstlisting}
while <expression>:
    <statement>
    [statement]
else:
    <statement>
    [statement]
\end{lstlisting}

\section{Exercises}

Given the code \ldots
\begin{lstlisting}
i = 1
while i 
\end{lstlisting}
\begin{enumerate}
	\item How many lines of output will the above code produce?
	\item What needs to be done to correct the program?
	\item Write a program that outputs the word `repeat' 100 times,
			each on a line of its own.
	\item Write a program that prints the numbers from 1 to 10 on the
			screen, on a single line, ending with a new line.
	\item Write a program that asks the user for a number and then
			prints the numbers from 1 to the number they entered.
	\item Write a program that asks the user for a number and then
			prints the sum of numbers from 1 to the number they
			entered.
	\item Write a program that asks the user for two numbers and then
			prints the sum of numbers from the lowest number entered to the
			highest number entered.
	\item Write a program that asks the user to enter a sequence of
			numbers, ending with a blank line. Print out the smallest of those
			numbers.
	\item Write a program that asks the user to enter a sequence of
			numbers, ending with a blank line. Print out the average of those
			numbers.
	\item Write a program that prints the numbers from 1 to 100, 10
			per each line.
\end{enumerate}

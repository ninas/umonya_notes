\documentclass[a4paper,11pt]{article}
\usepackage{ulem}
\usepackage{a4wide}
\usepackage[dvipsnames,svgnames]{xcolor}
\usepackage[pdftex]{graphicx}
\title{Introductory Programming in Python}
\usepackage[utf8]{inputenc}
\usepackage{hyperref}
% commands generated by html2latex


\begin{document}

\section{Introductory Programming in Python}

\section{Download Python}

You can download Python through the following   \href{http://python.org/ftp/python/2.6.4/python-2.6.4.msi}{link.

\section{Course Outline}
\begin{enumerate}
	\item \href{basic_concepts.html}{Introduction to Programming}
	\item \href{invocation.html}{Introduction to Python}
	\item \href{basic_output.html}{Basic Output}
	\item \href{basic_input.html}{Basic Input}
	\item \href{program_state.html}{Program State and Basic Variables}
	\item \href{conditionals.html}{Flow Control: Conditionals}
%  functions 

	\item \href{while_loops.html}{Flow Control: Conditional Loops}
	\item \href{lists.html}{Lists}
	\item \href{tuples.html}{Tuples}
	\item \href{for_loops.html}{Flow Control: Sequential Loops}
	\item \href{strings.html}{Strings in Depth}
	\item \href{functions.html}{Flow Control: Functions}
%   <li><a href="problems.html">Problem Solving Session</a></li>
%  
% 			<li><a href="dictionaries.html">Dictionaries</a></li>
% 
% 			<li><a href="scope.html">Variable Scope</a></li>
% 
% 			<li><a href="importing_modules.html">Importing Standard Modules</a></li>
% 
% 			<li><a href="commandline_arguments.html">More on Command Line Arguments</a></li>
% 
% 			<li><a href="random.html">Random Numbers</a></li>
% 
% 			<li><a href="files.html">Files for Input and Output</a></li>
% 
% 			<li><a href="regexp.html">Regular Expressions</a></li>
% 
% 			<li><a href="parsing.html">Basic Parsing</a></li>
% 
% 			<li><a href="os.html">Operating System Functionality</a></li>
% 
% 			<li><a href="datetime.html">Handling Dates and Times</a></li>
% 
% 			<li><a href="writing_errors.html">Writing Meaningful Error Messages</a></li>
% 
% 			<li><a href="understanding_errors.html">Understanding Python's Error Messages</a></li>
% 
% 			<li><a href="exceptions.html">Flow Control: Exceptions</a></li>
% 
% 			<li><a href="debugging.html">Debugging</a></li>
% 
% 			<li><a href="recursion.html">Flow Control: Recursion</a></li>
% 			
% 			<li><a href="classes.html">Classes and Objects</a></li>
% 
% 			<li><a href="queues.html">Advanced Data Structures: Queues</a></li>
% 
% 			<li><a href="stacks.html">Advanced Data Structures: Stacks</a></li>
% 
% 			<li><a href="trees.html">Advanced Data Structures: Trees</a></li>
% 
% 			<li><a href="database_theory.html">Database Theory</a></li>
% 			
% 			<li><a href="database_relational.html">Relational Databases</a></li>
% 			
% 			<li><a href="database_SQL.html">Structured Query Language</a></li>
% 			
% 			<li><a href="database_coding.html">Interfacing with databases using Python</a></li>
% 
% 			<li><a href="parallel_processing.html">Parallel Processing</a></li>
% 
% 			<li><a href="eventflowcontrol.html">Event Oriented Flow Control</a></li>
% 
% 			<li><a href="gui.html">Graphical User Interfaces with WxPython</a></li>
% 
% 			<li><a href="web.html">Web Programming and Web Services</a></li>
% 
% 		</ol>
% 
% 		<ul>
% 			<li><a href="assignments.html">Assignments</a></li>
% 		</ul>
% 


\section{Other Practice Material}

Practice makes perfect, although in terms of programming, it might be   better said that practice makes adequate. Learning to program   \textbf{requires} practice, and lots of it. While there are   exercises provided with each lesson, going from basic to tricky, there   are not nearly enough to make a good programmer from a complete   beginner. Truth be told, no number of exercises can do this. Becoming a   good programmer requires a desire to learn, and a desire to solve   problems, both classical problems in the form of exercises, and   problems of your own devising/encounter. That said, if you cannot find   enough of the former, here are some links to resources that provide   problems for you to solve.
\begin{itemize}
	\item \href{http://projecteuler.net/}{Project Euler}: A    comprehensive list of mathematically oriented problems. Excellent    practice for generic problem solving skills.
	\item \href{http://www.codingdojo.org}{The Coding Dojo}:    Presents the idea of a coding journal club of sorts. Check out the    KataCatalogue for problems to solve, or see if you can set something    like this up for yourself with buddies.
	\item \href{http://www.spoj.pl}{Sphere Online Judge}: Provides    a large set of problems, and you can upload your solution for    automated testing. Check out the 'problems' link on the menu on the    left.
\end{itemize}
% <h2>Miscellaneous Material</h2>
% 
% 		<ul>
% 			<li><a href="admin.html">Course Administration Miscellany</a></li>
% 
% 			<li><a href="vim.html">Vim Tutorial</a></li>
% 		</ul>
% 


\section{Reference Material}
\begin{itemize}
	\item \href{http://docs.python.org/index.html}{Python Documentation}
	\item \href{http://biopython.org}{BioPython Homepage}
	\item \href{http://www.bioinformatics.org/bradstuff/bp/tut/Tutorial.html}{An easy intro to BioPython}
	\item \href{http://www.dalkescientific.com/writings/NBN/}{Andrew Dalke's Pages}. Andrew taught advanced python for the NBN and a few extra courses in previous years. Excellent notes, very helpful and covers a wide variety of topics. His \href{http://www.dalkescientific.com/writings/diary/archive/2007/08/17/kata_and_practice.html}{'blog' post on coding katas} (a la coding dojos) provides some interesting reading.
	\item \href{http://ceee.rice.edu/Books/LA/}{A Superb Introduction to Linear Algebra}. Starts with absolute basics and builds up without tons of extraneous reading. Excellent as a refresher or if you've never done this stuff before.
\end{itemize}
\end{enumerate}   Copyright \copyright James Dominy 2007-2008; Released under the \href{http://www.gnu.org/copyleft/fdl.html}{GNU Free Documentation License}
\\\href{intropython.tar.gz}{Download the tarball}

\end{document}
